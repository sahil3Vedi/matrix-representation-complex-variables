\documentclass[12pt]{article}

\usepackage{amsmath}
\usepackage{amssymb}
\usepackage[pdftex]{graphicx}
\usepackage{multicol}

\date{November 2021}
\title{Matrix Representation of Complex Variables}
\author{Sahil Trivedi}

\begin{document}
    \maketitle

    \section{Introduction}
    Complex numbers are generally represented by the formula
    $$z = a + ib \mid (a,b) \in \mathbb{R}  $$
    However, they can also be represented as $2 X 2$ matrices.
    $$z = \begin{bmatrix}
        a & -b \\
        b & a 
    \end{bmatrix}$$
    Using this representation we eliminate the necessity of using $i$ to denote the direction orthogonal to the $real$ axis. The $real$ and $imaginary$ axes on the complex planes are base vectors in an $R^2$ space.
    Matrices of the form
    $$\begin{bmatrix}
        a & -b \\
        b & a 
    \end{bmatrix}  \mid (a,b) \in \mathbb{R} $$
    form a set over which addition and multiplication is commutative, and division is defined for matrices with a non-zero determinant. The only condition under which the determinant of a matrix of this form becomes $0$ is if $a=0$ and $b=0$. The determinant is the square of the distance of the complex number from the origin of the complex plane. 
    \\\\
    Besides addition, multiplication, and division, we can also define exponents over this group.






    \section{Logarithms of Complex Numbers}
    Because the group is defined over the set
    $$\begin{bmatrix}
        x & -y \\
        y & x 
    \end{bmatrix} \mid (x,y) \in \mathbb{R} $$
    A complex number raised to the power of another complex number can be denoted by
    $$ z_{1}^{z_{2}} = {\begin{bmatrix} a_{1} & -b_{1} \\ b_{1} & a_{1}   \end{bmatrix}}^{\begin{bmatrix} a_{2} & -b_{2} \\ b_{2} & a_{2}   \end{bmatrix}}$$
    In classical notation a complex number with unit magnitude can be described by
    $$ e^{i\theta} = \cos{\theta} + i\sin{\theta}$$
    In matrix representation, this is expressed as
    $$ {\begin{bmatrix}
        e & 0 \\
        0 & e 
    \end{bmatrix}}^{\begin{bmatrix}
        0 & -1 \\
        1 & 0 
    \end{bmatrix}\begin{bmatrix}
        \theta & 0 \\
        0 & \theta 
    \end{bmatrix}} = \begin{bmatrix}
        \cos{\theta} & {-\sin{\theta}} \\
        \sin{\theta} & \cos{\theta} 
    \end{bmatrix} $$
    $$\therefore {\begin{bmatrix}
        e & 0 \\
        0 & e 
    \end{bmatrix}}^{\begin{bmatrix}
        0 & -\theta \\
        \theta & 0 
    \end{bmatrix}} = \begin{bmatrix}
        \cos{\theta} & {-\sin{\theta}} \\
        \sin{\theta} & \cos{\theta} 
    \end{bmatrix} $$
    Taking the logarithm on both sides,
    $$ \begin{bmatrix}
        0 & -\theta \\
        \theta & 0 
    \end{bmatrix} = log(\begin{bmatrix}
        \cos{\theta} & {-\sin{\theta}} \\
        \sin{\theta} & \cos{\theta} 
    \end{bmatrix})
    $$
    % At $\theta = \pi/2$
    % $$
    % log(\begin{bmatrix}
    %     0 & -1 \\
    %     1 & 0} 
    % \end{bmatrix}) = \begin{bmatrix}
    %     \pi/2 & 0 \\
    %     0 & \pi/2
    % \end{bmatrix}
    % $$
    We can make good use of this result while calculating logarithms. For any complex number $z$:
    $$ z = re^{i\theta} = \begin{bmatrix}
        r & 0 \\
        0 & r 
    \end{bmatrix} {\begin{bmatrix}
        e & 0 \\
        0 & e 
    \end{bmatrix}} ^ {\begin{bmatrix}
        0 & -\theta \\
        \theta & 0
    \end{bmatrix}} = \begin{bmatrix}
        r & 0 \\
        0 & r
    \end{bmatrix} \begin{bmatrix}
        \cos{\theta} & {-\sin{\theta}} \\
        \sin{\theta} & \cos{\theta} 
    \end{bmatrix}$$
    We can represent $z$, $r$, and $\theta$ in terms of $a$ and $b$.
    $$ r = \sqrt{a^2 + b^2} $$
    $$ \cos{\theta} = \frac{a}{\sqrt{a^2 + b^2}} $$
    $$ \sin{\theta} = \frac{b}{\sqrt{a^2 + b^2}} $$
    $$ \therefore
    z = \begin{bmatrix}
        a & -b \\
        b & a 
    \end{bmatrix} = \sqrt{a^2 + b^2} \begin{bmatrix}
        \frac{a}{\sqrt{a^2 + b^2}} & \frac{-b}{\sqrt{a^2 + b^2}} \\
        \frac{b}{\sqrt{a^2 + b^2}} & \frac{a}{\sqrt{a^2 + b^2}}
    \end{bmatrix} 
    $$
    Based on this conversion can derive a general expression for calculating logarithms for complex numbers
    $$ log(z) = 
    log \left(
        \sqrt{a^2 + b^2} \begin{bmatrix}
            \frac{a}{\sqrt{a^2 + b^2}} & \frac{-b}{\sqrt{a^2 + b^2}} \\
            \frac{b}{\sqrt{a^2 + b^2}} & \frac{a}{\sqrt{a^2 + b^2}}
        \end{bmatrix} 
    \right)
    $$ 
    $$ \therefore log(z) = 
    \frac{1}{2}log(a^2 + b^2) +
    \begin{bmatrix}
        0 & -\sin^{-1}\left({\frac{b}{\sqrt{a^2 + b^2}}}\right) \\
        \sin^{-1}\left({\frac{b}{\sqrt{a^2 + b^2}}}\right) & 0
    \end{bmatrix} 
    $$
    $$ \therefore log(z) = log(\begin{bmatrix}
        a & -b \\
        b & a
    \end{bmatrix} ) = 
    \begin{bmatrix}
        \frac{1}{2}log(a^2 + b^2) & -\sin^{-1}\left({\frac{b}{\sqrt{a^2 + b^2}}}\right) \\
        \sin^{-1}\left({\frac{b}{\sqrt{a^2 + b^2}}}\right) & \frac{1}{2}log(a^2 + b^2)
    \end{bmatrix}
    $$

    \section{Exponents of Complex Numbers}
    Given this result we can finally calculate exponents for matrices
    $$ z_{1}^{z_{2}} = {\begin{bmatrix} a_{1} & -b_{1} \\ b_{1} & a_{1}   \end{bmatrix}}^{\begin{bmatrix} a_{2} & -b_{2} \\ b_{2} & a_{2}   \end{bmatrix}}$$
    $$ \therefore
    log(z_{1}^{z_{2}}) = \begin{bmatrix} a_{2} & -b_{2} \\ b_{2} & a_{2}   \end{bmatrix} log( \begin{bmatrix} a_{1} & -b_{1} \\ b_{1} & a_{1}   \end{bmatrix} )
    $$
    $$ \therefore
    log(z_{1}^{z_{2}}) = \begin{bmatrix} a_{2} & -b_{2} \\ b_{2} & a_{2} \end{bmatrix}
    \begin{bmatrix}
        \frac{1}{2}log({a_{1}}^2 + {b_{1}}^2) & -\sin^{-1}\left({\frac{b_1}{\sqrt{{a_1}^2 + {b_1}^2}}}\right) \\
        \sin^{-1}\left({\frac{b_1}{\sqrt{{a_1}^2 + {b_1}^2}}}\right) & \frac{1}{2}log({a_{1}}^2 + {b_{1}}^2)
    \end{bmatrix} 
    $$
    $$ \therefore log(z_{1}^{z_{2}}) = \begin{bmatrix} 
        \frac{a_2}{2}log({a_{1}}^2 + {b_{1}}^2) - {b_2}\sin^{-1}\left({\frac{b_1}{\sqrt{{a_1}^2 + {b_1}^2}}}\right) & - \frac{b_2}{2}log({a_{1}}^2 + {b_{1}}^2) - {a_2}\sin^{-1}\left({\frac{b_1}{\sqrt{{a_1}^2 + {b_1}^2}}}\right) \\ 
        \frac{b_2}{2}log({a_{1}}^2 + {b_{1}}^2) + {a_2}\sin^{-1}\left({\frac{b_1}{\sqrt{{a_1}^2 + {b_1}^2}}}\right) & \frac{a_2}{2}log({a_{1}}^2 + {b_{1}}^2) - {b_2}\sin^{-1}\left({\frac{b_1}{\sqrt{{a_1}^2 + {b_1}^2}}}\right)
    \end{bmatrix}
    $$
    We decompose this matrix into it's real and imaginary components and write the $ z_{1}^{z_{2}} $ in the exponential form, such that:
    $$ z_{1}^{z_{2}} = A \cdot B$$
    where
    $$ A = 
    e^{
        \begin{bmatrix} 
            \frac{a_2}{2}log({a_{1}}^2 + {b_{1}}^2) - {b_2}\sin^{-1}\left({\frac{b_1}{\sqrt{{a_1}^2 + {b_1}^2}}}\right) & 0 \\ 
            0 & \frac{a_2}{2}log({a_{1}}^2 + {b_{1}}^2) - {b_2}\sin^{-1}\left({\frac{b_1}{\sqrt{{a_1}^2 + {b_1}^2}}}\right)
        \end{bmatrix}
    } 
    $$
    $$ B = e^{
        \begin{bmatrix} 
            0 & - \frac{b_2}{2}log({a_{1}}^2 + {b_{1}}^2) - {a_2}\sin^{-1}\left({\frac{b_1}{\sqrt{{a_1}^2 + {b_1}^2}}}\right) \\ 
            \frac{b_2}{2}log({a_{1}}^2 + {b_{1}}^2) + {a_2}\sin^{-1}\left({\frac{b_1}{\sqrt{{a_1}^2 + {b_1}^2}}}\right) & 0
        \end{bmatrix}
    }
    $$
    $$ \therefore A = e^{\left(\frac{a_2}{2}log({a_{1}}^2 + {b_{1}}^2) - {b_2}\sin^{-1}\left({\frac{b_1}{\sqrt{{a_1}^2 + {b_1}^2}}}\right)\right)} $$
    $$ \therefore A = \sqrt{{\left({a_1}^{2} + {b_1}^{2}\right)}^{a_2}} \cdot e^{- {b_2}\sin^{-1}\left({\frac{b_1}{\sqrt{{a_1}^2 + {b_1}^2}}}\right)}$$
    $$ \therefore B = e^{\left(\left[
        \frac{b_2}{2}log({a_{1}}^2 + {b_{1}}^2) + {a_2}\sin^{-1}\left({\frac{b_1}{\sqrt{{a_1}^2 + {b_1}^2}}}\right)\right] \begin{bmatrix} 
            0 & - 1 \\ 
            1 & 0
        \end{bmatrix}
    \right)} $$
    $$ \therefore
    B = {\left(
        e^{
            \frac{b_2}{2}log({a_{1}}^2 + {b_{1}}^2) + {a_2}\sin^{-1}\left({\frac{b_1}{\sqrt{{a_1}^2 + {b_1}^2}}}\right)
        }
    \right)}^{\begin{bmatrix} 
        0 & - 1 \\ 
        1 & 0
    \end{bmatrix}}
    $$
    $$ \therefore B = 
    {\left(
        \sqrt{{\left({a_1}^{2} + {b_1}^{2}\right)}^{b_2}} \cdot
        e^{{a_2}\sin^{-1}\left({\frac{b_1}{\sqrt{{a_1}^2 + {b_1}^2}}}\right)}
    \right)}^{i}
    $$
    Finally, we get the value of $ z_{1}^{z_{2}} $ 
    $$ \therefore
    z_{1}^{z_{2}} = \sqrt{{\left({a_1}^{2} + {b_1}^{2}\right)}^{\left(a_{2} + ib_{2}\right)}} \cdot
    e^{\left((i{a_2} - b_{2})\sin^{-1}\left({\frac{b_1}{\sqrt{{a_1}^2 + {b_1}^2}}}\right)\right)}
    $$
    $$ \therefore
    z_{1}^{z_{2}} = \sqrt{{\left({a_1}^{2} + {b_1}^{2}\right)}^{\left(a_{2} + ib_{2}\right)}} \cdot
    e^{\left(i({a_2} + ib_{2})\sin^{-1}\left({\frac{b_1}{\sqrt{{a_1}^2 + {b_1}^2}}}\right)\right)}
    $$
    $$ \therefore
    z_{1}^{z_{2}} = \sqrt{{\left({a_1}^{2} + {b_1}^{2}\right)}^{\left(a_{2} + ib_{2}\right)}} \cdot
    {\left(e^{\left(i\sin^{-1}\left({\frac{b_1}{\sqrt{{a_1}^2 + {b_1}^2}}}\right)\right)}\right)}^{a_{2} + ib_{2}}
    $$
    $$ \therefore 
    z_{1}^{z_{2}} = \sqrt{{\left({a_1}^{2} + {b_1}^{2}\right)}^{\left(a_{2} + ib_{2}\right)}} \cdot
    {\left(\frac{a_{1}+ib_{1}}{\sqrt{{a_1}^2 + {b_1}^2}}\right)}^{a_{2} + ib_{2}}
    $$
    $$ \therefore
    z_{1}^{z_{2}} = {\left(a_1 + ib_1\right)}^{a_{2} + ib_{2}}
    $$

    \section{So, Is there a solution?}
    We do have a few techniques to find solutions of the form $z = a + ib$ for the value of $z_{1}^{z_{2}} = {\left(a_1 + ib_1\right)}^{a_{2} + ib_{2}}$.
    Depending on the values of $a_1 , b_1 , a_2$ and $b_2$:
    \subsection{a_1 = 0}
    $$ z_{1}^{z_{2}} = {\left(ib_1\right)}^{a_{2} + ib_{2}}$$
    $$ \therefore z_{1}^{z_{2}} = {i}^{a_2} \cdot {\left({i}^{i}\right)}^{b_2} \cdot {b_1}^{a_2} \cdot {\left({b_1}^{b_2}\right)}^{i}$$
    $ {i}^{a_2} $ may or may not have a  real solution \\\\
    $ {\left({i}^{i}\right)}^{b_2} $ is actually a real number because $ i^i $ has a real value (can you guess what it is?) \\\\
    $ {b_1}^{a_2} $ is a real number \\\\
    $ \left({b_1}^{b_2}\right)}^{i} $ may or may not have a real solution \\\\
    Also note that $ i^i $ has multiple solutions due to $\sin$ and $\cos$ being modular transforms.
    \subsection{b_1 = 0}
    $$ z_{1}^{z_{2}} = {\left(a_1\right)}^{a_{2} + ib_{2}}$$
    $$ \therefore z_{1}^{z_{2}} = {a_1}^{a_2} \cdot {\left({a_1}^{b_2}\right)}^{i} $$
    While ${a_1}^{a_2}$ is real ${\left({a_1}^{b_2}\right)}^{i}$ may or may not be real.
    \subsection{a_2 = 0}
    $$ z_{1}^{z_{2}} = {\left(a_1 + ib_{1}\right)}^{ib_2}$$
    We can convert the equation to an exponential form
    $$ z_{1}^{z_{2}} = {z_1}^{ib_2} = e^{ib_{2} \cdot log(z_1)}$$
    $$ \therefore z_{1}^{z_{2}} = \cos(ib_{2} \cdot log(z_1)) + i\sin(ib_{2} \cdot log(z_1))$$
    Both of these terms may or may not be real. There are cases where $log(z_1)$ cancels out with $i$ or $b_2$ and the solution becomes easier to calculate. 
    \\\\In any case you can always use the power series expansions of $e,sin(\theta)$ and $log(z)$ to compute the answers to the needed degree.
    \subsection{b_2 = 0}
    $$ z_{1}^{z_{2}} = {z_1}^{a_2}$$
    $$ \therefore z_{1}^{z_{2}} = e^{a_{2}\cdot log(z_2)} = e^{i \cdot -a_{2}i\cdot log(z_2)}$$
    $$ \therefore z_{1}^{z_{2}} = \cos(a_{2}i\cdot log(z_2)) - i\sin(a_{2}i\cdot log(z_2))$$
    \subsection{How computers perform matrix exponentiation}
    Computers mainly use the power series for functions like $e,sin(\theta)$ and $log(z)$.
    $$ \therefore {z_1}^{z_2} = e^{z_2 \cdot log(z_1)} = e^{i \cdot -z_{2}i \cdot log(z_1)}$$
    $$ \therefore {z_1}^{z_2} = \cos(z_{2}i \cdot log(z_1)) - i\sin(z_{2}i \cdot log(z_1))$$

    \section{Conclusion}
    We saw how complex numbers can be represented as matrices and how operations like logarithms and exponentiations can be achieved for complex numbers.
    
    
    
    
    
    % If we try to calculate the log of ${z_{1}}^{z_{2}}$, we get
    % $$ log({z_{1}}^{z_{2}}) = 
    % \begin{bmatrix}
    %     a_{2} & -b_{2} \\
    %     b_{2} & a_{2}
    % \end{bmatrix}
    % log(\begin{bmatrix}
    %     a_{1} & -b_{1} \\
    %     b_{1} & a_{1}
    % \end{bmatrix})
    % $$
    % $$
    % log(\begin{bmatrix}
    %     a_{1} & -b_{1} \\
    %     b_{1} & a_{1}
    % \end{bmatrix}) = 
    % log(\begin{bmatrix}
    %     a_{1} & 0 \\
    %     0 & a_{1}
    % \end{bmatrix} + \begin{bmatrix}
    %     0 & -b_{1} \\
    %     b_{1} & 0
    % \end{bmatrix})
    % $$







    % We define $r$ as the magnitude of $z_{1}$.
    % $$ r = \sqrt{{a_{1}^{2} + {b_{1}}^{2}}}$$
    % $$\therefore log({z_{1}}^{z_{2}}) = \begin{bmatrix}
    %     a_{2} & -b_{2} \\
    %     b_{2} & a_{2}
    % \end{bmatrix}
    % log(\begin{bmatrix}
    %     r & 0 \\
    %     0 & r
    % \end{bmatrix} \begin{bmatrix}
    %     \cos{\theta} & {-\sin{\theta}} \\
    %     \sin{\theta} & \cos{\theta} 
    % \end{bmatrix})$$
    % $$\therefore log({z_{1}}^{z_{2}}) = \begin{bmatrix}
    %     a_{2} & -b_{2} \\
    %     b_{2} & a_{2}
    % \end{bmatrix} \left[
    %     log(\begin{bmatrix}
    %         r & 0 \\
    %         0 & r
    %     \end{bmatrix})
    %     +
    %     log(\begin{bmatrix}
    %         \cos{\theta} & -\sin{\theta} \\
    %         \sin{\theta} & \cos{\theta}
    %     \end{bmatrix})
    % \right] $$
    % $$ \therefore log({z_{1}}^{z_{2}}) = \begin{bmatrix}
    %     a_{2} & -b_{2} \\
    %     b_{2} & a_{2}
    % \end{bmatrix} \left(
    %     \begin{bmatrix}
    %         log(r) & 0 \\
    %         0 & log(r)
    %     \end{bmatrix}
    %     +
    %     \begin{bmatrix}
    %         0 & -\theta \\
    %         \theta & 0
    %     \end{bmatrix}
    % \right) $$
    % $$ 
    % \therefore log({z_{1}}^{z_{2}}) = \begin{bmatrix}
    %     a_{2} & -b_{2} \\
    %     b_{2} & a_{2}
    % \end{bmatrix}  \begin{bmatrix}
    %     log(r) & -\theta \\
    %     \theta & log(r)
    % \end{bmatrix}
    % $$
    % $$
    % \therefore log({z_{1}}^{z_{2}}) = \begin{bmatrix}
    %     a_{2}log(r) - b_{2}\theta & -a_{2}\theta - b_{2}log(r) \\
    %     a_{2}\theta + b_{2}log(r) & a_{2}log(r) - b_{2}\theta
    % \end{bmatrix}   
    % $$
    % $$
    % {z_{1}}^{z_{2}} = {\begin{bmatrix}
    %     e & 0 \\
    %     0 & e
    % \end{bmatrix}}^{\begin{bmatrix}
    %     a_{2}log(r) - b_{2}\theta & -a_{2}\theta - b_{2}log(r) \\
    %     a_{2}\theta + b_{2}log(r) & a_{2}log(r) - b_{2}\theta
    % \end{bmatrix}}
    % $$
    % We separate the orthogonal components inside the exponents
    % $$
    % {z_{1}}^{z_{2}} = {\begin{bmatrix}
    %     e & 0 \\
    %     0 & e
    % \end{bmatrix}}^{\left(\begin{bmatrix}
    %     a_{2}log(r) - b_{2}\theta & 0 \\
    %     0 & a_{2}log(r) - b_{2}\theta
    % \end{bmatrix} + \begin{bmatrix}
    %     0 & -a_{2}\theta - b_{2}log(r) \\
    %     a_{2}\theta + b_{2}log(r) & 0
    % \end{bmatrix}\right)}
    % $$
    % $$ \therefore
    % {z_{1}}^{z_{2}} = {\begin{bmatrix}
    %     e & 0 \\
    %     0 & e
    % \end{bmatrix}}^{\begin{bmatrix}
    %     a_{2}log(r) - b_{2}\theta & 0 \\
    %     0 & a_{2}log(r) - b_{2}\theta
    % \end{bmatrix}} {\begin{bmatrix}
    %     e & 0 \\
    %     0 & e
    % \end{bmatrix}}^{\begin{bmatrix}
    %     0 & -a_{2}\theta - b_{2}log(r) \\
    %     a_{2}\theta + b_{2}log(r) & 0
    % \end{bmatrix}}
    % $$
    % Assuming $a_{2}\theta + b_{2}log(r) = K$
    % $$
    % {z_{1}}^{z_{2}} = \begin{bmatrix}
    %     \cos{K} & -\sin{K} \\
    %     \sin{K} & \cos{K}
    % \end{bmatrix}{\begin{bmatrix}
    %     e & 0 \\
    %     0 & e
    % \end{bmatrix}}^{\begin{bmatrix}
    %     a_{2}log(r) - b_{2}\theta & 0 \\
    %     0 & a_{2}log(r) - b_{2}\theta
    % \end{bmatrix}} 
    % $$
    % $$
    % \therefore 
    % {z_{1}}^{z_{2}} = \frac{\begin{bmatrix}
    %     \cos{K} & -\sin{K} \\
    %     \sin{K} & \cos{K}
    % \end{bmatrix}{\begin{bmatrix}
    %     e & 0 \\
    %     0 & e
    % \end{bmatrix}}^{\begin{bmatrix}
    %     a_{2}log(r) & 0 \\
    %     0 & a_{2}log(r)
    % \end{bmatrix}}}{{\begin{bmatrix}
    %     e & 0 \\
    %     0 & e
    % \end{bmatrix}}^{\begin{bmatrix}
    %     b_{2}\theta & 0 \\
    %     0 & b_{2}\theta
    % \end{bmatrix}}}
    % $$
    % $$
    % \therefore {z_{1}}^{z_{2}} = r^{a_{2}}\frac{\begin{bmatrix}
    %     \cos{K} & -\sin{K} \\
    %     \sin{K} & \cos{K}
    % \end{bmatrix}}{{\begin{bmatrix}
    %     e & 0 \\
    %     0 & e
    % \end{bmatrix}}^{\begin{bmatrix}
    %     b_{2}\theta & 0 \\
    %     0 & b_{2}\theta
    % \end{bmatrix}}}
    % $$
    % $$ \therefore {z_{1}}^{z_{2}} =
    % r^{a_{2}} e^{b_{2}\theta} \begin{bmatrix}
    %     \cos{K} & -\sin{K} \\
    %     \sin{K} & \cos{K}
    % \end{bmatrix}
    % $$
    % Next we need to evaluate the value of $K$. Recall that
    % $$
    % K = a_{2}\theta + b_{2}log(r)
    % $$
    % $$
    % \therefore \cos{K} = \cos(a_{2}\theta + b_{2}log(r))
    % $$$$
    % \therefore \cos{K} = \cos({a_{2}\theta})\cos({b_{2}log(r)}) - \sin({a_{2}\theta})\sin({b_{2}log(r)})
    % $$
    % and
    % $$
    % \sin{K} = \sin{a_{2}\theta}\cos{b_{2}log(r)} + \sin{b_{2}log(r)}\cos{a_{2}\theta}
    % $$

\end{document}